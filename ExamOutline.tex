\documentclass{amsart}

\title{Exam Overview \\ MAS332 Complex Analysis}

\begin{document}

The exam has had an extremely similar format the previous years.  

\section*{Question 1}
Essentially all review from first and second years about basic manipulations of complex numbers and about line integrals.  Main questions:

\begin{enumerate}
\item Write a given complex number in form $x+iy$ or $re^{i\theta}$
\item State and use the triangle inequalities
\item Find solution(s) to an equation of complex numbers involving polynomials, log, exp, and trig functions
  \item Compute a line integral
  \end{enumerate}

Some line integrals will be zero just by citing cauchy's theorem, but there will almost always be one where this doesn't apply and you need to do it ``by hand''.

\section*{Question 2}
A little bit of a grab-bag.  Mostly focused on what it means for a function to be differentiable at apoint / analytic and applications of the Cauchy-Riemann equations, but a little bit more about line integrals sometimes appears.

\begin{enumerate}
\item Defining a region, what it means for a function of a complex variable to be differentiable / analaytic, finding where it is differentiable / analytic
\item Stating Cauchy-Riemann equations and basic applications, e.g., finding all analytic functions with a given real/imaginary part
  \item Line integrals: Using ``ML bounds'' and theorems about antiderivatives
  \end{enumerate}

\section*{Question 3}
Very tight focus on this question -- Cauchy's theorems and applications
\begin{enumerate}
\item State Cauchy's Theroem and Cauchy's integral formula(s)
  \item Use Cauchy's theorem to calculate contour integrals
\end{enumerate}  

Usually will be some contour integrals where Cauchy's formula doesn't apply, just to make sure you understand the conditions required.

\section*{Question 4}

\end{document}
