 \documentclass{beamer}
\beamertemplatenavigationsymbolsempty
\usepackage{amsmath, amssymb, hyperref, graphics, tikz}
\usepackage[normalem]{ulem} %for strikeout \sout{ }

\usepackage{tikzsymbols} %for smileys

\newcommand{\C}{\mathbb{C}}
\newcommand{\Z}{\mathbb{Z}}
\newcommand{\R}{\mathbb{R}}
\newcommand{\N}{\mathbb{N}}
\DeclareMathOperator{\Real}{Re}
\DeclareMathOperator{\Imag}{Im}
\begin{document}
\begin{frame}{Previously: Cauchy's Integral Formula and Consequences}
\begin{theorem}[Cauchy's integral formula] Let $\gamma$ be a simple contour described in the positive direction.  Let $w$ lie inside $\gamma$.  Suppose that $f$ is analytic on a simply connected region $D$ containing $\gamma$ and its interior.  Then:

$$f(w)=\frac{1}{2\pi i} \int_\gamma\frac{f(z)}{z-w}dz.$$

\end{theorem}
\begin{block}{Uses of Cauchy's Integral Formula}
\begin{itemize}
    \item Calculate contour integrals
    \item Values of $f$ inside a contour determined by values on boundary
    \item \alert{Proving $f$ has convergent Taylor series!}
\end{itemize}
\end{block}
\begin{block}{Today: Other applications}
\end{block}

\end{frame}


\begin{frame}{Analytic functions have all derivatives!}
\begin{theorem}[Cauchy's Integral formula for the derivatives]
Let $\gamma$ be a simple contour described in the positive direction.  let $w$ be any point inside $\gamma$.  Suppose $f$ analytic on a simply-connected region $D$ containing $\gamma$.
Then $$f^{(n)}(w)=\frac{n!}{2\pi i}\int_\gamma \frac{f(z)}{(z-w)^{n+1}}dz$$

\end{theorem}

\begin{proof}Take $\frac{d}{dw}$ of both sides of CIF.  Differentiate inside the integral.
\end{proof}
\begin{example} Let $\gamma$ be the square with vertices $1, i, -1,-i$.  Evaluate
$$\int_\gamma\frac{e^z}{z^n}dz$$

\end{example}
\end{frame}

\begin{frame}{Another application of CIF: Liouville's Theorem}
\begin{theorem}[Liouville's Theorem]
A function which is analytic and bounded in the complex plane is a constant.
\end{theorem}
\begin{proof}Let $a, b\in \C$.
\begin{enumerate}
    \item Rewrite $f(a)-f(b)$ as an integral around $|z|=R$ using CIF.  
    \item Use ML to bound $|f(a)-f(b)|$
    \item As $R\to\infty$, the bound goes to 0.
\end{enumerate}    
\end{proof}

Toy applications of Liouville's Theorem frequently on exam, and in problem sheets.  More excitingly, Liouville's Theorem can prove the fundamental theorem of algebra.
\end{frame}

\begin{frame}{Hope you're excited as I am}
\begin{center}
    \includegraphics[width=\textwidth,height=0.8\textheight,keepaspectratio]{FundamentalTheoremSuccess.jpg}
\end{center}


\end{frame}


\begin{frame}{A theorem you've long used...}
\begin{theorem}[Fundamental Theorem of algebra] Let $p(z)$ be a non-constant polynomial with complex coefficients.  Then there is a point $w\in\C$ such that $p(w)=0.$\end{theorem}

\begin{proof}
Suppose not, and $p(z)$ has no roots. Then show $1/p(z)$ is bounded and analytic on $\C$, and apply Liouville's Theorem.
\end{proof}
Using induction, it follows that a polynomial of degree $n$ has $n$ roots, counted with multiplicity.

\begin{block}{Note}
The trick of dividing by a nonzero function appears frequently in applications.
\end{block}

\end{frame}


\end{document}
