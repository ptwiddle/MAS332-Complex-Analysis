\documentclass{beamer}
\beamertemplatenavigationsymbolsempty
\usepackage{amsmath, amssymb, hyperref, graphics, tikz}
\usepackage[normalem]{ulem} %for strikeout \sout{ }

\usepackage{tikzsymbols} %for smileys

\newcommand{\C}{\mathbb{C}}
\newcommand{\Z}{\mathbb{Z}}
\newcommand{\R}{\mathbb{R}}
\newcommand{\N}{\mathbb{N}}
\DeclareMathOperator{\Real}{Re}
\DeclareMathOperator{\Imag}{Im}
\DeclareMathOperator{\Res}{Res}
\begin{document}



\begin{frame}{Classification of Singularities}
By Laurent's Theorem, if $f(z)$ is analytic in a punctured disk around $\alpha$, it has a convergent Laurent expansion
$$f(z)=\sum_{n\in\Z} a_n (z-\alpha)^n$$
\begin{block}{Three possibilities:}
\begin{description}
   \item[Removable singularity] None of the $a_n$ with $n<0$ are nonzero
       \item[A pole] Only finitely many $a_n$ with $n<0$ are nonzero
    \item[Essential singularity] Infinitely many $a_n$ with $n<0$ are nonzero

\end{description}
\end{block}

\begin{block}{Typical Questio (first step to using Residue Theorem):}
Find and classify the singularities of $f(z)$, and find the residue at each one.
\end{block}

\end{frame}


\begin{frame}{Examples:}

  
   $$\frac{1}{1+z^2} \quad\quad\quad \frac{1}{e^z-1}\quad\quad\quad z\cos\left(\frac{1}{z-1}\right)$$
    $$\frac{\cos(z)}{z^2\sin(z)} \quad\quad\quad \frac{\tan{z}}{z}$$  

      
  


\end{frame}  

\begin{frame}{Easy (and examinable!) theorems about poles}
\begin{theorem} $f$ has a pole of order $k$ at $\alpha$ if and only if $$f(z)=\frac{g(z)}{(z-\alpha)^k}$$ where $g(z)$ analytic and nonzero in some disk around $\alpha$.
\end{theorem}

\begin{theorem} If $f$ has a zero of order $k$ at $\alpha$, then $1/f$ has a pole of order $k$ at $\alpha$.
\end{theorem}

\begin{corollary}If $f$ has a zero of order $m$ at $\alpha$, and $g$ has a zero of order $n$ at $\alpha$, then
\begin{itemize}
    \item $\frac{f}{g}$ has a pole of order $n-m$ if $n>m$
    \item $\frac{f}{g}$ has a removable singularity if $m\geq n$
\end{itemize}
\end{corollary}
\end{frame}

\begin{frame}{Easy way to find residues at poles}
\begin{theorem}
Suppose that $f$ has a pole of order $k$ at $\alpha$.  Then

$$\Res\{f;\alpha\}=\frac{1}{(k-1)!}\lim_{z\to\alpha} \frac{d^{k-1}}{dz^{k-1}} (z-\alpha)^kf(z)$$
\end{theorem}
\begin{proof}Just compute the right hand side.
\end{proof}
\begin{corollary}
If $f=g/h$, where $g$ and $h$ are analytic at $\alpha$, $g(\alpha)\neq 0, h(\alpha)=0, h(\alpha)\neq 0$, then $f$ has a simple pole at $\alpha$ and
$$\Res\{f;\alpha\}=\frac{g(\alpha)}{h^\prime(\alpha)}$$
\end{corollary}
\end{frame}

\end{document}

