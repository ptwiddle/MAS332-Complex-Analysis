\documentclass{beamer}
\beamertemplatenavigationsymbolsempty
\usepackage{amsmath, amssymb, hyperref, graphics, tikz}
\usepackage[normalem]{ulem} %for strikeout \sout{ }

\usepackage{tikzsymbols} %for smileys

\newcommand{\C}{\mathbb{C}}
\newcommand{\Z}{\mathbb{Z}}
\newcommand{\R}{\mathbb{R}}
\newcommand{\N}{\mathbb{N}}
\DeclareMathOperator{\Real}{Re}
\DeclareMathOperator{\Imag}{Im}
\DeclareMathOperator{\Res}{Res}
\begin{document}

\begin{frame}{Last time: Basics of Laurent series}
A Laurent series is like a power series but we're allowed to have negative terms:
$$\sum_{n=-\infty}^\infty a_n(z-a)^n$$

\begin{theorem}[Laurent's Theorem] Suppose that $f$ has an isolated singularity at $\alpha$, so $f$ analytic on $D^\prime=\{z: 0< |z-\alpha| < R\}$.  Then $f$ can be represented by a Laurent series around $\alpha$ that converges on $D^\prime$:
$$f(z)=\sum_{n=-\infty}^\infty a_n (z-\alpha)^n$$
Where:
$$a_n=\frac{1}{2\pi i} \int_{C_r(\alpha)} \frac{f(w)}{(w-\alpha)^{n+1}}dw$$
\end{theorem}

\end{frame}

\begin{frame}{What's left to cover:}
\begin{description}
\item[11.5+11.6] Classification of singularities
\item[11.7+11.8] Calculating residues at poles
\item[12+12.1] Residue Theorem
\item[12.2] Application of Residue Theorem to real integrals
\end{description}

\begin{definition}[Residue] Let $f$ have an isolated singularity at $\alpha$, and  Laurent expansion:
$$f(z)=\sum_{n=-\infty}^\infty a_n (z-\alpha)^n$$
Then $a_{-1}$ is called \emph{The residue of $f$ at $\alpha$}, and written $\Res\{f;\alpha\}.$
\end{definition}
\begin{block}{Awkward ordering in notes:}
The end of 11 is really about finding Residues, but we only care about these because of the Residue Theorem.
\end{block}

\end{frame}


\begin{frame}{To fix this, changing order of lectures}
To motivate the material at the end of Section 11, going to cover the Residue Theorem first.

\begin{description}
\item[Today] Residue Theorem: Proof + first examples
\item[Thursday] Classifying Singularities + finding residues
\item[Next Tuesday] Applying Residue Theorem to real integrals
\item[Next Thursday] Revision; focus on last two weeks
\end{description}

The material next lecture will make applying Residue Theorem easier in nice cases.


\end{frame}

\begin{frame}{}
\begin{theorem}[The Residue Theorem]
Let $D$ be a simply connected region containing a simple positively oriented contour $\gamma$.  Suppose $f$ is analytic on $D$ except for finitely many singularities $\beta_1,\dots, \beta_n$, none of which like on $\gamma$.  Then

$$\int_\gamma f(z)dz=2\pi i \times (\text{sum of the residues of $f$ at the $\beta_i$ inside $\gamma$})$$

\end{theorem}
\begin{proof}
The proof is really putting together things we've already done:
\begin{itemize}
    \item Deform contour so one singularity in each piece
    \item Expand $f$ in Laurent series
    \item Use formula for $a_{-1}$ / our first important example
\end{itemize}
\end{proof}
\end{frame}


\begin{frame}{}
\begin{columns}
\begin{column}{0.5\textwidth}
    \includegraphics[width=\textwidth,height=0.8\textheight,keepaspectratio]{YodaTea.jpeg}
\end{column}



\begin{column}{0.5\textwidth}
\sout{Every mother on Christmas morning}

Every lecturer when they prove the big theorem of the module

\end{column}
\end{columns}
\end{frame}
\begin{frame}{Using the Residue Theorem}
Show you understand and \alert{check hypotheses}!

\begin{enumerate}
    \item Find the bad points (isolated singularities) $\beta_i$ of $f$ 
    \item Draw picture showing $\gamma$ and bad points to see which are inside
    \item Find the residues at the bad points inside $\gamma$
\end{enumerate}
\begin{block}{Examples; let  $c=5e^{it}\quad (0\leq t\leq 2\pi)$}
\begin{enumerate}
\item $\int_c\frac{dz}{z^2(z-3)^3}$
\item $\int_c \frac{dz}{\tan(z)}$
\item $\int_c z^3\cos(1/z)dz$
\end{enumerate}
\end{block}
\begin{block}{Take-away:}
Using Residue Theorem from definition can be slightly painful; can we find residue without finding whole Laurent expansion?
\end{block}


\end{frame}





\end{document}
