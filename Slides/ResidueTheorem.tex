\documentclass{beamer}
\beamertemplatenavigationsymbolsempty
\usepackage{amsmath, amssymb, hyperref, graphics, tikz}
\usepackage[normalem]{ulem} %for strikeout \sout{ }

\usepackage{tikzsymbols} %for smileys

\newcommand{\C}{\mathbb{C}}
\newcommand{\Z}{\mathbb{Z}}
\newcommand{\R}{\mathbb{R}}
\newcommand{\N}{\mathbb{N}}
\DeclareMathOperator{\Real}{Re}
\DeclareMathOperator{\Imag}{Im}
\DeclareMathOperator{\Res}{Res}
\begin{document}

\begin{frame}{Last week: Laurent series, residues, singularities}
A Laurent series is like a power series but we're allowed to have negative terms.
\begin{theorem}[Laurent's Theorem] Suppose that $f$ has an isolated singularity at $\alpha$, so $f$ analytic on $D^\prime=\{z: 0< |z-\alpha| < R\}$.  Then $f$ can be represented by a Laurent series around $\alpha$ that converges on $D^\prime$:
$$f(z)=\sum_{n=-\infty}^\infty a_n (z-\alpha)^n$$
\end{theorem}
\begin{itemize}
\item $a_{1}$ is called the \emph{residue of $f$ at $\alpha$}
\item $\int_{C_r(\alpha)}f(z)dz=a_{-1}$ for small $r$
\item $f$ has a removable singularity/pole/essential singularity if it has no/finite / infinite $a_{-k}\neq 0$
\end{itemize}





\end{frame} 

\begin{frame}{What's left:}
\begin{description}
\item[Today] Residue Theorem
\item[Tomorrow] Applying Residue Theorem to Real Integrals
\item[Next Monday] Residue Theorem tricks: $$\sum_{n=1}^\infty \frac{1}{n^2}=\frac{\pi^2}{6}$$
\item[Tuesday] Application: Laplace Transform?  Revision?
\item[Week 12] Revision?
\end{description}

No office hours this week due to the strike.
\end{frame}


\begin{frame}{At last!}

\begin{theorem}[The Residue Theorem]
Let $D$ be a simply connected region containing a simple positively oriented contour $\gamma$.  Suppose $f$ is analytic on $D$ except for finitely many singularities $\beta_1,\dots, \beta_n$, none of which like on $\gamma$.  Then

$$\int_\gamma f(z)dz=2\pi i \times (\text{sum of the residues of $f$ at the $\beta_i$ inside $\gamma$})$$

\end{theorem}



\begin{proof}
The proof is really putting together things we've already done:
\begin{itemize}
    \item Deform contour so one singularity in each piece
    \item Expand $f$ in Laurent series
    \item Use formula for $a_{-1}$ / our first important example
\end{itemize}
\end{proof}
\end{frame}


\begin{frame}{Dated meme, but we put our tree up this weekend}
\begin{columns}
\begin{column}{0.5\textwidth}
    \includegraphics[width=\textwidth,height=0.8\textheight,keepaspectratio]{YodaTea.jpeg}
\end{column}



\begin{column}{0.5\textwidth}
\sout{Every mother on Christmas morning}

Every lecturer when they prove the big theorem of the module

\end{column}
\end{columns}
\end{frame}
\begin{frame}{Using the Residue Theorem}
Show you understand and \alert{check hypotheses}!

\begin{enumerate}
    \item Find the isolated singularities (bad points) $\beta_i$ of $f$ 
    \item Draw picture showing $\gamma$ and bad points to see which are inside
    \item Find the residues of the singularities inside $\gamma$
\end{enumerate}
\begin{block}{Examples; let  $c=5e^{it}\quad (0\leq t\leq 2\pi)$}
\begin{enumerate}
\item $\int_c\frac{dz}{z^2(z-3)^3}$
\item $\int_c \frac{dz}{\tan(z)}$
\item $\int_c z^3\cos(1/z)dz$
\end{enumerate}
\end{block}
\begin{block}{Finding the residues can be a lot of work.}
Laurent Expansion often easiest. For simple poles use shortcut.
\end{block}


\end{frame}





\end{document}
