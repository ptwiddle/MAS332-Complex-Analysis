 \documentclass{beamer}
\beamertemplatenavigationsymbolsempty
\usepackage{amsmath, amssymb, hyperref, graphics, tikz}
\usepackage[normalem]{ulem} %for strikeout \sout{ }

\usepackage{tikzsymbols} %for smileys

\newcommand{\C}{\mathbb{C}}
\newcommand{\Z}{\mathbb{Z}}
\newcommand{\R}{\mathbb{R}}
\newcommand{\N}{\mathbb{N}}
\DeclareMathOperator{\Real}{Re}
\DeclareMathOperator{\Imag}{Im}
\begin{document}


\begin{frame}{He's \emph{not} a real smooth function}
\begin{columns}
\begin{column}{0.6\textwidth}
    \includegraphics[width=\textwidth,height=0.8\textheight,keepaspectratio]{AlwaysTaylor.jpg}
\end{column}
\begin{column}{0.4\textwidth}
The following function from $\R\to\R$ is infinitely differentiable everywhere, but the Taylor series at 0 converges nowhere:
$$f(x)=\begin{cases} 0 & x\leq 0 \\
e^{-1/x} & x>0 \end{cases}$$
\end{column}
\end{columns}

\end{frame}



\begin{frame}{Analytic Functions have Taylor Series}
\begin{theorem} Let $f$ be analytic on $\Delta=\{z : |z-z_0|<r\}$, where $r>0$.  Then $f$ has a Taylor expansion about $z_0$ that is valid on all of $\Delta$: 
$$f(z)=\sum_{n=0}^\infty \frac{f^{(n)}(z_0)}{n!}(z-z_0)^n$$
\end{theorem}
\begin{proof}
CIF +  Expand $1/(w-z)$ as geometric series in $(z-z_0)/(w-z_0)$\\
\alert{Warning:} Usual role of $w$ and $z$ in CIF reversed for this proof.
\end{proof}

\begin{block}{Cool corollary}
The radius of convergence around $z_0$ is the distance to the first point $w$ where $f$ isn't analytic.

In particular, if $f$ is entire (analytic on all of $\C$), then its Taylor expansions converge everywhere!
\end{block}
\end{frame}

\begin{frame}{Calculating Taylor series}
\begin{block}{When possible, avoid taking derivatives}
Too much work.
\end{block}

\begin{block}{Instead...}
\begin{itemize}
    \item Build from Taylor series you know: $1/(1-z), \exp, \sin,\dots$
    \item Helps psychologically to substitute $w=z-z_0$
    \item Can differentiate / integrate Taylor series term by term
\end{itemize}
\end{block}
\begin{block}{Examples: find Taylor series of...}
\begin{enumerate}
    \item $1/(z+1)$ around $z=2$
    \item $z^3\cosh(z^2)$ around $z=0$
    \item $1/(1-z)^2$ around $z=0$ 
    \end{enumerate}

\end{block}


\end{frame}

\begin{frame}{Zeroes of functions}
\begin{definition} Let $f$ be analytic on a region $D$.  We say $w\in D$ is a \emph{zero of $f$} if $f(w)=0$.  We say $w$ is a \emph{zero of order $k$} if $f(w)=f^\prime(w)=\cdots=f^{(k-1)}(w)=0$, but $f^{(k)}(w)\neq 0$.
\end{definition}
\begin{example}
$z\sin(z)$ has a zero of order 2 at 0, and a zero of order 1 at $k\pi$ for $0\neq k\in\Z$
\end{example}
\begin{lemma}
 $f(z)$ has a zero of order $k$ at $a$ if and only if $f(z)=(z-a)^kg(z)$, where $g(z)$ is analytic on nonzero on an open set containing $a$.
\end{lemma}
\begin{corollary}
If $f(z)$ has a zero of order $m$ at $a$, and $g(z)$ has a zero of order $n$ at $a$, then $f\cdot g$ has a zero of order $m+n$ at $k$.
\end{corollary}

\end{frame}


\end{document}
