\documentclass{beamer}
\beamertemplatenavigationsymbolsempty
\usepackage{amsmath, amssymb, hyperref, graphics, tikz}
\usepackage[normalem]{ulem} %for strikeout \sout{ }

\usepackage{tikzsymbols} %for smileys

\newcommand{\C}{\mathbb{C}}
\newcommand{\Z}{\mathbb{Z}}
\newcommand{\R}{\mathbb{R}}
\newcommand{\N}{\mathbb{N}}
\DeclareMathOperator{\Real}{Re}
\DeclareMathOperator{\Imag}{Im}
\DeclareMathOperator{\Res}{Res}
\begin{document}

\begin{frame}{Revision: Classification of singularities + finding residues}
\begin{theorem}
Suppose that $f$ has a pole of order $k$ at $\alpha$.  Then

$$\Res\{f;\alpha\}=\frac{1}{(k-1)!}\lim_{z\to\alpha} \frac{d^{k-1}}{dz^{k-1}} (z-\alpha)^kf(z)$$
\end{theorem}
\begin{corollary}
If $f=g/h$, where $g$ and $h$ are analytic at $\alpha$, $g(\alpha)\neq 0, h(\alpha)=0, h(\alpha)\neq 0$, then $f$ has a simple pole at $\alpha$ and
$$\Res\{f;\alpha\}=\frac{g(\alpha)}{h^\prime(\alpha)}$$
\end{corollary}
\begin{block}{Examples: Classify singularities; find residues}
$$(1+z)^2e^{i/z} \qquad \frac{z^2}{\sinh(\pi z)} \qquad \frac{1}{(1+z+z^2)^4}$$
\end{block}
\end{frame}
\begin{frame}{Computing $\int_{-\infty}^\infty f(x)dx$ with the Residue Theorem}

\begin{block}{Basic idea is very flexible:}
$$\int_{-\infty}^\infty f(x)dx=\lim_{R,S\to\infty}\int_{-R}^S f(x)dx$$
\begin{itemize}
    \item  Include the finite integral as part of a contour integral
    \item Calculate the contour integral using residue theorem
    \item As $R,S\to\infty$ contributions of other parts of contour $\to 0$
    \end{itemize}
\end{block}
\begin{block}{ML Estimates \emph{sometimes}  work for last point}
Basic example / sanity check: 
$$\int_{-\infty}^\infty \frac{1}{x^2+1}dx=\pi$$
\end{block}
\end{frame}

\begin{frame}{Specific formulation appearing in notes on exam}
I don't really like that you just need to memorize this, but \Sey[2][yellow]
\begin{theorem}
Let $f(z)=\frac{p(z)}{q(z)}e^{i\lambda z}$
where $\lambda\in\R, \lambda>0$, $p(z),q(z)$ are polynomials with $\deg q>\deg p$, no common zeroes, and $q$ has no real roots.  Then
$$\int_{-\infty}^\infty \frac{p(x)}{q(x)} e^{i\lambda x} dx=\sum_{i=1}^k\Res\{f;z_k\}$$
where $z_1, z_2,\dots, z_k$ are the zeros of $q$ in the upper half plane.
\end{theorem}
\begin{proof}
Follow plan from last slide; if $\deg(q)+1<\deg(p)$ can just use $ML$-estimates, otherwise we need to sweat a bit more.
\end{proof}


\end{frame}

\begin{frame}{Applying our specific formulation to compute real integrals}

\begin{block}{Take real and imaginary parts of our integrand}
If $p,q$ have real coefficients, and $x$ is real, then
$$\Real \frac{p(x)}{q(x)}e^{i\lambda x}=\frac{p(x)}{q(x)}\cos(\lambda x)\quad \Imag \frac{p(x)}{q(x)}e^{i\lambda x}=\frac{p(x)}{q(x)}\sin(\lambda x)$$
\end{block}
\begin{block}{Examples from Section 12.3 of Notes}
$$\int_{-\infty}^\infty \frac{x\sin(\pi x)}{x^2+2x+5}dx=-\pi e^{-2\pi}$$
$$\int_0^\infty \frac{\cos(\pi x)}{(1+x^2)^2}dx=\frac{\pi(\pi+1)e^{-\pi}}{4}$$
$$\int_{0}^\infty \frac{\cos^2(x)}{1+x^2}dx=\frac{\pi(1+e^2)}{4e^2}$$





\end{block}
\end{frame}
\begin{frame}{Thank you!  Thursday will be revision}
\center
\includegraphics[width=\textwidth,height=.9\textheight,keepaspectratio]{ItsOver.jpg}

\end{frame}

\end{document}
